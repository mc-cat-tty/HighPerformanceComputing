\chapter{Introduction}

Introduction on heterogenous system, parallel programming models, hardware accelerators, etc.

\section{Resources}
Slides are generally enough + lab hands-on.

Further readings:
\begin{itemize}
  \item An Introduction to Parallel Programming, Peter Pacheco
  \item CUDA reference manual
\end{itemize}

\section{Course Overview, Prerequisites and Objectives}
\subsection{Objectives}
We are going to see:
\begin{itemize}
  \item architectural paradigms for parallel and accelerated computation such as multicore CPU, GPU, FPGA, and NPU
  \item program profiling and performance evaluation, along with issues and optimization opportunities
  \item Intermediate OpenMP programming for Intel CPU programming
  \item Intermediate CUDA programming for NVIDIA GPU programming
  \item HLS - High Level Synthesis - for FPGA programming
\end{itemize}

\subsection{Overview}
Course units:
\begin{enumerate}
  \item Theoretical unit about parallel and heterogenous computing architectures and common hardware models
  \item Programming models for parallel architectures with hands-on labs: NVIDIA Tegra and Xilinx Zyqn architecture have been picked as platforms for lab sessions
\end{enumerate}

We should rather talk about GPGPU, which stands for General Purpose GPU, since nowadays they are used improperly for tasks other than graphics rendering.

Be aware of the difference between dGPU and iGPU:
\begin{itemize}
  \item iGPU - Integrated GPU - less powerful but data can be moved faster
  \item dGPU - Discrete GPU - much powerful but you pay the price of traversing a path far from the CPU
\end{itemize}

Development boards enable introspection on the system during development process: we can understand what's going on in the system.

\subsection{Motivation}
Supercomputing vs High Performance Computing. They are different: HPC moves the focus on SoC (not huge scale computers) and their heterogeneous/parallel composition.
The same challenges that where thightly bound to supercomputers are now exhibited by this kind of smaller pocket systems.

Embedded Systems (ESs) are the central actors of this revolution. An escalation in this sector has been marked by the following concepts:
\begin{itemize}
  \item 
\end{itemize}

See Gordon Bell, Renesas R-CAR V3U, Hilisicon Kirin

Final objective: learn how to offload workload from main CPU to dedicated subsystems.

See big.LITTLE ARM architecture